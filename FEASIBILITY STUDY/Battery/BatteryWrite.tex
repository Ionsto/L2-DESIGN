\documentclass{article}
\usepackage{amsmath}
\usepackage{graphicx}
\usepackage{float}
\begin{document}
\section{Overview}
Dense energy storage is a difficult task, and batteries are a must for pure electric vehicles.
\section{Capacity}
The larger the capacity of a battery network, the larger, heavier, and more expensive the battery network becomes.
The use case of the system needs to be considered for choosing a capacity.
Capacitance is normally measured in Amp-Hours, or how many hours a battery can supply the rated current at the nominal voltage of the battery.

When specifying capacity for an electric vehicle, two factors are going to be needed:
\begin{itemize}
    \item Range
    \item Efficiency
\end{itemize}
The nominal range will be determined by the expected average daily usage, and the specified charge frequency. 
The value given for charge frequency will be once a week.
The daily usage will be a round trip from Josephine butler to the engineering department on the science site each way this is ~1.5km, this is shown in Fig. \ref{fig:route}.
\begin{figure}[H]
    \centering
    \scalebox{0.5}{\includegraphics{./RouteAnalysis/Route.png}}
    \caption{Nominal daily route}
    \label{fig:route}
\end{figure}
The exact elevation change can be seen in \ref{fig:route_el}, with the gradients of this data in \ref{fig:route_grad}.
This data is taken from GPS public databases via https://www.gpsvisualizer.com/elevation.
The routes were made using google maps - exported to GPS route format.
\begin{figure}[H]
    \centering
    \scalebox{0.5}{\includegraphics{./RouteAnalysis/elevation.png}}
    \caption{Gradient of route}
    \label{fig:route_el}
\end{figure}
Fig. \ref{fig:route_el} shows the elevation through the route, with mountjoy easily identifiable. 
\begin{figure}[H]
    \centering
    \scalebox{0.5}{\includegraphics{./RouteAnalysis/grad.png}}
    \caption{Elevation of route}
    \label{fig:route_grad}
\end{figure}
There is a total of 50 meters uphill, and 25 meters downhill when going towards butler.
The effect of the height change on the required battery capacity will be non negligible.
If we model the energy requirement as a simple vehicle hoist, the energy will be $mg\Delta h$.
The total vertical distance is 75m.
Roughly 90kJ more energy will be needed with a perfectly efficient system.
Since a joule is one watt second, 90kJ is 25 watt hours.

Rough power estimates: 
10 watt hours per mile - electric bike
12 watt hours per journey (flat)

The total Watt hours of the journey including the vertical distance is 37 watt hours.

For a 5 journeys a week
$37 \cdot 5 = 185Wh$
185 watt hours is needed to make the journey from 100\% battery to flat.



\section{Charging}
Battery charging is difficult, specific charging circuits are needed for most modern high density batteries.
When trying to charge lots of batteries quickly a large amount of heat is generated, which needs to be considered in the design stage.
Large modern batteries can also be very dangerous - and this is amplified when charging, so care needs to be taken that the design isn't a health hazard.
\subsection{Lifespan}
The lifespan of a battery is determined by the depth of discharge, and the family of battery used.
\begin{figure}[H]
    \centering
    \scalebox{0.5}{\includegraphics{./ChargeCycles/DoD.png}}
    \caption{Lifespan with respect to usage}
    \label{fig:route}
\end{figure}
Lithium ion batteries have a cycle life of about 800 cycles at 50\% DoD, whereas Lithium polymer does much better at approximately 2000 charges at 50\%.
For this reason we are using LiPo batteries as they offer the better battery life.
Given the capacity to 100\% DoD is 185Wh, the final capacity will need to be 370Wh.
\subsection{Rate of charge}
The rate of charging is a nonlinear function with the charge of the battery, therefore it doesn't to quote charge time from 0\%-100\%.
It makes sense to specify a rate of recharge from a specific depth of discharge.
The specific depth of discharge will be determined by the lifespan of the battery, and we can consider a "charge cycle" charging from DoD\% to 95\%.

%https://www.dnkpower.com/lithium-polymer-battery-guide/
%https://batteryuniversity.com/learn/article/how_to_prolong_lithium_based_batteries
\section{Battery size}
The battery size and weight are limiting factors on capacity.
There are two parts to battery 'size' the volume, and mass of the battery are critical factors for a moving vehicle.
\begin{figure}[H]
    \centering
    \scalebox{0.5}{\includegraphics{./Density.png}}
    \caption{Credits: NASA - National Aeronautics and Space Administration}
    \label{fig:route}
\end{figure}
\end{document}
