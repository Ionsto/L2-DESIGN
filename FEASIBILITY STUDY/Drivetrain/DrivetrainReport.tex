\documentclass{article}
\usepackage{amsmath}
\usepackage{graphicx}
\usepackage{float}
\begin{document}
\section{Introduction}
With any motor system there has to be a trade off between the speed of rotation and the torque output of the drive system In the case of this system there need to be a large maximum speed when on flat terrain and the required torque to overcome restive forces due to gravity, air resistance, friction and rolling resistance.

The simplest solution would be to directly attach motors to the wheel but with typical motor speed of ??RPM this would give a speed of the skateboard of ??? there is clearly a disparity here that needs to be fixed using some sort of drivetrain

\section{Types}
There are numerous types of drive train possible but this repot will mainly focus on:
\begin{itemize}
	\item Belt
	\item Gearing
	\item Hub motor
	\item Chain
\end{itemize}
\subsection{Belt}
Belt driven systems have a large amount of elasticity so they provide a low noise and vibration power transmission that is beneficial in this application as constant vibration from transmission would provide an unpleasant user experience.
Belt systems are also low maintenance and as they require no oiling and typically have very long life spans.
\subsection{Gearing}
Having the output of the motor go directly into a gear arrangement or a gear box would allow for a vast array of gear ratios as well as the potential to change gears in the case of certain gear boxes however for this application this is likely to be unnecessary due to the typically large range of output speeds of DC motors. Another potential advantage of using gearing is that it allows for simple arrangements with on motor driving 2 wheels
\subsection{Chain}
Using a chain to drive the board is the most efficient way to transmit power out of the options covered. 
However it is high maintenance as due to the chain being metallic there is need for lubrication to prevent excess wear and oxidation. 
\end{document}

