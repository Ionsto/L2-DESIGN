\documentclass[journal,10pt]{IEEEtran}
\usepackage{amsmath}
\begin{document}
\section{Overview}
	Any complex electronic system will almost always require a control unit to operate it. This can take many forms, from off-the-shelf solutions to bespoke circuits.
	There are two aspects of control that must be designed for; the user interface to control the skateboard, battery management, and motor/drive train control.
\section{User Interface}
	\subsection{Overview}
		\begin{itemize}
			\item Need a way to control the motion of the board
			\item Need to relay information back to the rider in a non distracting way
			\item Need to make charging experience as easy as possible -> no special experience required.
		\end{itemize}
	\subsection{Throttle Control}
		The user needs to be able to control the speed of the board intuitively and quickly, to ensure safe riding.
		A potential solution to this is a dual pressure pad system, where leaning forward increases forward throttle, and leaning backwards increases braking/reverse.
		There are already pre-existing market leader (boosted use a wireless controller with a roller as a throttle control. 
		This allows for both accelerating, breaking, and reverse, with step wise control for fine adjustment.
		This only takes effect when a secondary button is held, acting as a dead-mans-switch.
		Might not be prudent to have direct throttle control of the system, perhaps a fly by wire interface?
	\subsection{Battery Meter}
		The user needs to be able to understand what the remaining range of the board is, so that they will not undertake a journey where they will run out of charge mid-journey.
\section{Battery Management}
	\subsection{Overview}
		\begin{itemize}
			\item Safe charging and discharging 
			\item Safe current draw during normal operation
			\item Monitoring battery health including charge cycle number
		\end{itemize}
	\subsection{Charging}
		Lithium Polymer batteries have a very high energy density, which although desirable, leads to a range of safety concerns, firstly how to charge them without causing a fire.
		General guidelines suggest charging at no faster than 1C for optimal battery lifetime, meaning a 1300maH battery is charged at 1.3A.
		The number of cycles must be counted, as well as the current battery temperature to ensure that the battery is in a healthy state whenever in use.
		This includes preventing charging when the battery has reached the end of its operational lifetime.
		If regenerative braking is pursued, the circuitry must be carefully controlled so that the charging current is not exceeding these values.
	\subsection{Discharging}
		Lithium Polymer batteries can be discharged at a much higher rate than their charging limitations, with typical values for constant discharge in the 40-50C range, with a bust rating (no longer than 10 seconds) in the 90-120C range.
		The battery management system should therefore monitor the current draw from the battery, ensuring that the circuit is not operating outside these values.
\section{Motor Control}
	\subsection{Overview}
		\begin{itemize}
			\item Types of motor that could be used, brushed vs brushless
			\item control methods for each type, pwm vs 3 phase control
			\item Efficiency of each method
			\item power draw
			\item complexity
		\end{itemize}
	\subsection{Motor control}
		Since a DC Brushless motor is being used, a more complex control method is required than a brushed DC motor.
		Operating on a 3-phase AC supply, speed is varied by adjusting the frequency of the AC phases.
		This circuitry can be obtained off the shelf as an 'Electronic Speed Controller' and are widely available in a range of specifications due to their use in the remote control model industry.
		Control of these circuits is by pulse width modulation, or serial interface, varying by model and age of controller.
		This circuit could be designed and produced by hand, however due to the complexity, and the competitive pricing of the currently available products, it may prove a more effective choice to purchase an existing ESC.
	\subsection{Feedback}
		text goes here
\end{document}
