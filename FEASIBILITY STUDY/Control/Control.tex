\documentclass[journal,10pt]{IEEEtran}
\usepackage{amsmath}
\begin{document}
\section{Overview}
	Any complex electronic system will almost always require a control unit to operate it. This can take many forms, from off-the-shelf solutions to bespoke circuits.
	There are two aspects of control that must be designed for; the user interface to control the skateboard, battery management, and motor/drive train control.
\section{User Interface}
	\subsection{Overview}
		\begin{itemize}
			\item Need a way to control the motion of the board
			\item Need to relay information back to the rider in a non distracting way
			\item Need to make charging experience as easy as possible -> no special experience required.
		\end{itemize}
	\subsection{Throttle Control}
		The user needs to be able to control the speed of the board intuitively and quickly, to ensure safe riding.
		There are already pre-existing electric skateboards that use a wireless controller with a roller as a throttle control. 
		This allows for both accelerating, breaking, and reverse, with step wise control for fine adjustment.
		This only takes effect when a secondary button is held, acting as a dead-mans-switch.
		Might not be prudent to have direct throttle control of the system, perhaps a fly by wire interface?
	\subsection{Battery Meter}
		The user needs to be able to understand what the remaining range of the board is, so that they will not undertake a journey where they will run out of charge mid-journey.
\section{Battery Management}
	\subsection{Overview}
		\begin{itemize}
			\item Safe charging and discharging 
			\item Safe current draw during normal operation
			\item Monitoring battery health including charge cycle number
		\end{itemize}
	\subsection{Charging}
		Lithium Polymer batteries have a very high energy density, which although desirable, leads to a range of safety concerns, firstly how to charge them without causing a fire.
	\subsection{Discharging}
		text goes here
\section{Motor Control}
	\subsection{Overview}
		\begin{itemize}
			\item Types of motor that could be used, brushed vs brush-less
			\item control methods for each type, PWM vs 3 phase control
			\item Efficiency of each method
			\item power draw
			\item complexity
			\item Regenerative breaking
		\end{itemize}
	\subsection{Motor control}
		text goes here
	\subsection{Feedback}
		text goes here
\end{document}
